\chapter{Defining and Assembling Components with \module{peak.binding}}

\section{Component-Based Applications}

What's in a component, anyway?  Why use them to develop software?  Software
developers have dreamed for decades of a future where applications could be
built by simply plugging together off-the-shelf components.  In some
development environments, this is at least partly reality today.  Many GUI
programming tools let you construct at least the visual parts of an 
application by assembling components.

The promised benefits of component-based development architectures include
reusability (and therefore less repetitive work), reliability (if each part
works separately, and they are assembled correctly, the whole assembly should
work), and ease of understanding/maintenance (because parts can be understood
separately).

To be useful, a component architecture must include ways of:

\begin{itemize}

\item connecting components to form an application,

\item packaging and distributing the components, and

\item separating the work of an application into components.

\end{itemize}

Let's look at how PEAK addresses these issues.











\subsection{Composing vs. Connecting}

Imagine a car.  It's composed from a variety of parts: the wheels, engine, 
battery, frame, and so on.  Some of these parts are also composed of parts:
the engine has a block, cylinders, pistons, spark plugs, and so on.

Each part in this ``component assembly" can be a part of only one larger part:
its \strong{parent component}.  The hubcaps are part of the wheels,
and so they can't also be part of the engine.  (They wouldn't fit there in 
any case, but that's beside the point.)  Consider that screws or bolts may
be used in many parts of the car: each is part of only one other part of
the car, although more than one of the same \emph{kind} of part may be used
in other places.  In the UML (Unified Modelling Language) and in PEAK, this
kind of parent-child ``assembly" relationship  is called \strong{composition}:
a component is being ``composed" by assembling other components.

But in the UML and in real life, this isn't the only way of building things
with components.  It would be very inefficient if every light and accessory
in your car had to have its own, independent electrical system.  Ways of
\emph{sharing} components are needed.  In the car, wires, pipes, hoses, and
shafts serve to \emph{connect} the services provided by shared components to
the places where they are needed.  Note that such connections may be between
components at any level: wires carry electricity to every electrical part, no
matter how big or small.  In some cases, wires go to a major subsystem, which
then has internal wires to carry electricity inward to its parts, or to carry
signals between its parts.  In  the UML, these kind of ``shared" or
``peer-to-peer" connections are called \strong{associations}.














\subsubsection{Implementing Components in Python and PEAK}

In the Python language, components are Python objects, and composition and
association relationships are represented using objects' attributes.  Using
the \module{peak.binding} package, you'll create \strong{attribute bindings}
that  define what sub-objects will be created (via composition) or external
objects  will be referenced (via association) by each attribute of a component.

Of course, there are some important differences between software and the real
world.  In the real world, we have to actually build every part of the car
``ahead of time", and we must have one screw for every place a screw is needed.
In software, classes let us define the concept of a ``screw" once and then use
it as many times as we want, anywhere that we want.

Also, with PEAK, bindings are ``lazy".  What this means is that we can define
an ``engine" class whose parts aren't actually created until they're needed.
We just list the parts that are needed and what attributes they'll be bound
to, and when the attribute is used, the part is automatically created or
connected, according to our definition.

Since each part ``magically" appears the first time we want to use it, it's
as though it was always there.  It's as if your car was an empty shell until
you opened the door or looked in the window, at which point all the contents
magically appeared.  And then when you got into the car, the radio was just
an empty shell until you tried to turn it on, at which point all of its
internal components sprang into being and wired themselves together.

This ``lazy" construction technique can speed startup times for applications
which are built from large numbers of components, by not creating all the
objects right away, and by never creating objects that don't get used during
that application run.










\subsubsection{Component Interfaces}

You can't just hook wires between random parts of your car and expect good
results.  In the same way, software components can only be ``wired" together
if they have compatible interfaces.  A software component that expects to
send data to a ``file" component must be connected to a component that
provides the same services a file would provide, even if the component is not
actually a ``real" disk file.

A specific set of services that a component provides is called an
\strong{interface}.  Interfaces can denote a component's requirements, as well
as the guarantees that it provides when those requirements are met.  In PEAK,
interfaces are defined and declared using the Zope \module{Interface} package
(bundled with PEAK for your convenience).   This means that components you
create with PEAK's component architecture should also work in Zope 3's
component architecture.  (Which is important if you plan to build Zope
3-based applications and web services with PEAK.)

Interfaces in PEAK are used primarily as documentation and as a way of finding
compatible components and services.  When used with the Zope 3 component
architecture, interfaces can also be used to register adapters (which convert
a component from one interface to another), declare web views of application
components, and even define security restrictions based on interfaces.  

\vfill
\begin{seealso}\begin{itemize}

\item \citetitle{A Quick Introduction to Python Interfaces} at 
\url{http://www.zope.org/Wikis/Interfaces/InterfaceUserDocumentation}.  

\item And, if you're interested in developing Zope 3 applications and want to
learn more about what you can do with interfaces in the Zope 3 component
architecture, see \citetitle{Programming with the Zope 3 Component
Architecture} at \url{http://dev.zope.org/Wikis/DevSite/Projects/ComponentArchitecture/Zope3PythonProgrammerTutorialChapter1/
}.

\end{itemize}\end{seealso}
\vfill



\subsection{Applications = Components + Bindings}

For now, we're going to skip over the issue of packaging and distributing
components.  Since they're implemented as Python objects, we can use virtually
any of the standard techniques for packaging and distributing Python code,
such as the \module{distutils} package, or perhaps more elaborate systems such
as Gordon MacMillan's cross-platform \citetitle{Installer} package.

So let's move on to the third major aspect of a component architecture:
separating the work of an application or system into components.


\subsubsection{Why compose and connect?}

Is it realistic to expect to be able to define an application entirely in
terms of components?  And why would you want to do it in the first place?
Can't we just write an application the ``old-fashioned way"?  That is,
import the exact components we want, and use them wherever and whenever
we want, instead of creating bindings to link them to a master ``application"
component?

Well, you could, but one of PEAK's goals is to improve reusability.  Consider
this: all cars have engines, but different models of car have different goals
or requirements for their engines.  If we are creating a ``Car" application,
wouldn't it be nice if we could switch out the ``engine" component when we
get a new project, without having to maintain multiple versions of the code?
Perhaps for this project we need the ``car" to be compact, or perhaps we need
a bigger engine for a station wagon this time.

How can we achieve such reusability?  It actually requires only a simple
coding convention: never write code in a function or method that directly
references another class.  Instead, define \emph{all} collaborations with
other classes by way of instance attributes.  This means that collaborating
component classes can be easily substituted in a derived class, substituting
a new ``engine" class, for example.






While you can apply this technique of ``hiding collaborators" in any
object-oriented language, PEAK takes the approach to a new level.  Because
PEAK attribute bindings can be used to programmatically define connections in
context, components can actually seek out their collaborators dynamically at
runtime, via configuration files or other sources.  This can be as simple
and ubiquitous as looking up what database to connect to, or as complex as
specifying strategy components to select algorithms that are optimal
for a specific deployment environment.

We're back to the ``ilities" again: reusability, composability, extensibility,
and flexibility, in this case.  (Maybe portability, too.)  We even get a bit
of understandability and maintainability: when we isolate collaborations in
attribute bindings, it becomes a lot easier to implement the \citetitle{Law
of Demeter} and write ``adaptive programs".

\vfill

\begin{seealso}

\url{http://www.ccs.neu.edu/home/lieber/LoD.html} has lots of links about the 
\citetitle{Law of Demeter}, and you can also see
\url{http://www.ccs.neu.edu/research/demeter/demeter-method/LawOfDemeter/object-formulation.html
} for its ``object-oriented" version, if you'd like to know more about this
software quality technique.  PEAK takes this ``Law" very seriously, insisting
that code which references even a collaborator class must do so via an
instance attribute or other ``neighbor" as defined by the Law.

You don't have to know or follow the \citetitle{Law of Demeter} to use PEAK.
(Your programs just won't be as flexible.)  But all of our example programs 
will obey the Law, so you'll have a chance to see how -- and how easy it 
is -- to follow it.

\end{seealso}

\vfill






\subsubsection{Services, Elements, and Features}

So what's an application made of?  The PEAK approach describes application
components in terms of their lifecycles and roles, as follows:

\begin{description}
\item[Services] \hfill \\ 
Similar in concept to ``singletons", Services are ``well-known" instance objects
which exist for the lifetime of the application.  For example, a database
connection object in an application could be a service component, and so could
a top-level window in a GUI application.  J2EE ``session beans" and
``message-driven beans" are also good examples of service components.

\item[Elements] \hfill \\ 
Elements are typically ``problem-domain" objects or their proxies in
the user interface.  They are created and destroyed by other Elements or
by the application's Services.  Typical Elements might be ``business objects"
such as customers and sales, but of course any object that is the actual
subject of the application's purpose would be considered an Element.  In a
mail filtering program, for example, e-mail messages, mailboxes, and
``sender whitelist" objects would be considered Elements.

\item[Features] \hfill \\
Features are ``solution-domain" objects used to compose Elements, and less
often, to compose Services.  Features are typically used to represent 
the properties, methods, associations, user interface views, database fields,
and other ``features" of a problem-domain Element.  Feature objects are often
used as a class attribute in an Element's class, and shared between instances.
PEAK makes extensive use of feature objects to implement labor-saving
techniques such as generative programming.

\end{description}

This breakdown of application components is called the
\strong{Service-Element-Feature (SEF) Pattern}.  PEAK makes it easy to create 
components of each kind, but this pattern certainly isn't limited to PEAK.
You'll find Services, Elements, and Features in almost any object-based
application, regardless of language or platform.  But, the pattern is often
implemented in a rather haphazard fashion, and without the benefit of explicit
``wiring" between components.

When you design an application using the SEF Pattern, your top-level 
application object is itself a Service.  You construct that service from
lower-level services, such as database connections, object managers,
and maybe even a top-level window object for a GUI application.  If your
application is to be web-based, or you're constructing a middle-tier
subsystem, perhaps it will be composed of web services or the equivalent
of J2EE session beans.

The top-level application object may provide various methods or ``value-added"
services on top of its subcomponents' services, or it may just serve to start
them up in an environment that defines their mutual collaborators.  You may
find later that what you thought was an ``application" component, is really 
just another service that you want to use as part of a larger application.
Fortunately, it's easy to change a PEAK component's bindings and
incorporate it into a larger system.

Defining the Elements of your application design is also fairly
straightforward.  Elements are the subject of what your application
\emph{does}.  A business application might have Customer elements, a web
server might handle Page elements, and a hard drive utility might manipulate
Partition elements.

Features, on the other hand, are usually provided by frameworks, and
incorporated into your application's Elements to create a specific kind of
application.  For example, a GUI framework might provide visual features that
allow mapping Element properties to input fields in a window.

While PEAK does or will provide many Element and Feature base classes you can
use to build your applications, these facilities are outside the scope of
this tutorial, which focuses primarily on assembling an application's Service
components.  However, many of the techniques you'll learn here will be as
applicable to Element and Feature objects as they are to Service components.

So, let's get started on actually \emph{using} PEAK to build some
components, shall we?






\section{Specifying Attributes Using Bindings}

\subsection{Binding Fundamentals}

It's time to write some code!  Our objective: create a ``car" class 
that keeps track of its passengers.

\begin{verbatim}
>>> from peak.api import binding
>>> class Car:
        passengers = binding.New(dict, 'passengers')
	
>>> aCar=Car()
>>> print aCar.passengers
{}

\end{verbatim}

Let's go through this sample line by line.  In the first line we import the
\module{peak.binding} API.  Then, we create a simple class, with one attribute,
\member{passengers}.  We define the attribute using the \function{binding.New}
function, which creates an attribute binding from a type and an optional name.

Instances of class \class{Car} will each get their own \member{passengers}
attribute, which contains a new (hence the name, \function{binding.New})
dictionary.  If you're new to Python, you might wonder why we don't place a
dictionary directly in the class, like this:

\begin{verbatim}
>>> class Car:
        passengers = {}

\end{verbatim}

The problem is that attributes defined in a Python class body are shared, so
every car will end up with the same passenger dictionary.  Experienced Python
programmers solve this problem by placing code in their \method{__init__}
method to initialize the structure, like so:



\begin{verbatim}
>>> class Car:
        def __init__(self):
            self.passengers = {}

\end{verbatim}

This works alright for simple situations, but gets more complex when you use
inheritance to create subclasses of \class{Car}.  It becomes necessary to
call the superclass \method{__init__} methods, and the order of initialization
for different attributes can get tricky.

If you develop in Java or C++ or some other language that has instance
variable initializers, you'll be happy to know that PEAK lets you have them
in Python too -- only better.  In most static languages, variable initializers
run at object creation time.  So, either you create \emph{all} of an object's
collaborators at creation time, or you write accessor functions to hide
whether the fields or attributes are initialized.  This can be quite tedious
in complex programs.

But PEAK's attribute bindings are \strong{lazy}.  They do not compute their
value until they are used.  If we take a new instance of our \class{Car}
class, and print its dictionary before and after referencing the 
\member{passengers} attribute:

\begin{verbatim}
>>> anotherCar=Car()
>>> print anotherCar.__dict__
{}
>>> print anotherCar.passengers
{}
>>> print anotherCar.__dict__
{'passengers': {}}

\end{verbatim}

We find that the object doesn't really \emph{have} the attribute until we
try to access it, but once we do access it, it springs into being as though
it had always been there, and it acts like a normal attribute thereafter.


If you're familiar with the Eiffel programming language, you'll notice that
PEAK attribute bindings are quite similar to Eiffel's \strong{once functions},
except that they're for instances rather than for classes.  A ``once function"
is computed at most once, the first time its value is referenced.  In PEAK,
virtually all attribute bindings are based on \class{binding.Once}, a class
that handles most of the mechanics needed to make these ``once attributes" work.

\subsubsection{Deriving from \class{binding.Base}}

As we've already seen, attribute bindings can be made to work with 
``old-style" or ``classic" Python classes.  There are some drawbacks to that,
however.  Most visibly, it's necessary to specify an attribute binding's
name in its definition, as we saw in our example, i.e. 
\samp{passengers = binding.New(dict, 'passengers')}.

However, if we derive our class from \class{binding.Base}, we no longer have
to do this:

\begin{verbatim}
>>> class Car(binding.Base):
        passengers = binding.New(dict)
	
>>> aCar=Car()
>>> print aCar.passengers
{}

\end{verbatim}

Now, it's sufficient to use \samp{binding.New(dict)} to define the attribute.
What would happen if we did this \emph{without} \class{binding.Base}?











\begin{verbatim}
>>> class Car:
        passengers = binding.New(dict)
	
>>> aCar=Car()
>>> print aCar.passengers
Traceback (most recent call last):
  File "<pyshell#15>", line 1, in ?
    print aCar.passengers
  File "C:\PYTHON22\Lib\site-packages\peak\binding\once.py", line 137, in __get__
    self.usageError()
  File "C:\PYTHON22\Lib\site-packages\peak\binding\once.py", line 151, in usageError
    raise TypeError(
TypeError: <peak.binding.once.Once object at 0x01491EE0> was used in a type which
    does not support ActiveDescriptors, but a valid attribute name was not
    supplied

\end{verbatim}

Ouch!  It doesn't work, because PEAK can't tell what name the attribute
has without help from either you or the base class.  (Actually, it's the
metaclass, not the base class, but that's not important right now).  There
are some circumstances where PEAK will try to guess the attribute name for you,
such as when you use a class or a  function to define an attribute binding, but
for the most part you must either have a base class (such as 
\class{binding.Base}) whose metaclass derives from
\class{binding.ActiveDescriptors}, or else you must supply the attribute name
yourself when defining the binding.

So, if you plan to use attribute bindings in your program, it's probably best
to subclass \class{binding.Base}; it'll save you a lot of typing!

There are other \module{peak.binding} base classes which you'll use that
derive from \class{binding.Base}, though, like \class{binding.Component} and
\class{binding.AutoCreated}.  You'll use these latter two a lot when defining
Service components.  \class{binding.Base} is designed to be a minimalist
component base, without a lot of the extra features found in its subclasses.




\subsubsection{\class{binding.Once} - The Basis for all Bindings}

All PEAK bindings are created using the class \class{binding.Once}, or a
subclass thereof.  It provides the basic machinery needed to compute an
attribute "on-the-fly".  Here's a simple example of its use:

\begin{verbatim}
class Car(binding.Base):

    def height(self,instDict,attrName):
        baseHeight = self.roof.top - self.chassis.bottom
        return self.wheels.radius + baseHeight

    height = binding.Once(height)

\end{verbatim}

In this example, we want to compute the car's height based on various other
attributes, but we only want to compute it once, and save the value thereafter.
To do this, we define a function that takes three arguments: the object, its
instance dictionary, and the name of the attribute being computed.  Most of the
time, you'll only care about the first argument, which is the object for which
the attribute is being computed.  It's rare that you'll need the instance
dictionary or the attribute name, but if you need them, they're there.
Normally, you'll just return the value you want the attribute to end up with.

Note, by the way, that there's no need for this function passed to
\class{binding.Once} to be a method, or for the parameters to have specific
names.  If we continued the class above as follows:

\begin{verbatim}
    verticalCenter = binding.Once(lambda s,d,a: s.height/2)

\end{verbatim}

This would be a perfectly valid way to express the idea that the car's vertical
center is half of its height.  It's also true that we could say:




\begin{verbatim}
    passengers = binding.Once(lambda s,d,a: {})

\end{verbatim}

as another way of saying \samp{binding.New(dict)}.  However, the
\function{binding.New} function is more compact, and clearer as to intention.
Of course, its implementation consists of creating a function and returning a
\class{binding.Once} instance wrapped around that function, just the same as
we could do ``by hand".

By the way, although we've been only showing uses of \class{binding.Once} with
one argument (the function to be called), \class{binding.Once} does actually
take other arguments, such as a default attribute name, a ``provides" 
specification (more on this later) and a docstring.  Check out the PEAK
API Reference or the source code for more details.

























\subsubsection{Re-binding, Pre-binding, Un-binding, and Persistence}

We've mentioned over and over that attribute bindings are computed only once,
but that's not precisely true.  It's actually possible for them to be computed
zero times, or many times, if we set or delete the attribute the binding
defines.

You may have wondered, ``how does a binding know whether it's been computed or
not?"  It knows because there is an entry in the object's dictionary for that
attribute name.  (Notice, by the way, that this means objects without
dictionaries can't get much use out of bindings.)

But what if there's something already in the dictionary?  Or what if you remove
the value from the dictionary?  Well, it's pretty much as you'd expect.  If you
set a value for an attribute, then the binding will not compute a value.  If
you delete the attribute, and try to access it, the binding will recompute the
value.  This behavior can actually be quite useful in the context of 
transactions: many PEAK transactional components delete certain bindings at the
conclusion of a transaction, allowing them to be recomputed in the next
transaction.

PEAK also makes use of the ability to override a binding by manually setting
a value for an attribute.  For example, most PEAK component classes'
\method{__init__} methods accept keyword arguments which can override attributes
which would otherwise be determined by bindings.  This is especially useful in
conjunction with the \class{binding.requireBinding} class, which simply
raises an error when you try to access the attribute.  It's a handy way of
specifying that a component must have the attribute, and that a subclass or
instance must supply the value.

By the way, it's important to note that bindings do \emph{not} participate in
persistence, as they directly alter the associated object's dictionary,
bypassing the normal ``set attribute" machinery.  You should carefully
consider the use of attribute bindings in persistent objects, as to how they
will interact with your persistence machinery (e.g. whether they should be
saved or not) and whether you will want to code them in such a way as to
consider the object ``changed" when an attribute is computed.




\subsection{Creating Attribute Bindings}

\subsubsection{Binding API conventions}

name, provides, doc

\subsubsection{Basic Binding Classes/Functions}

\begin{classdesc}{Once}{func, name=None, provides=None, doc=None}
\end{classdesc}

\begin{funcdesc}{New}{obtype, bindToOwner=None, name=None, provides=None,
doc=None}
\end{funcdesc}

\begin{funcdesc}{Copy}{obj, name=None, provides=None, doc=None}
\end{funcdesc}

\begin{classdesc}{requireBinding}{description="", provides=None, doc=None}
\end{classdesc}

\begin{funcdesc}{bindToSelf}{provides=None, doc=None}
\end{funcdesc}

XXX AutoCreated, OnceClass


\section{Composing Hierarchies with \class{binding.Component} Objects}

XXX bindToParent


\section{Connecting Components by Name or Interface}

XXX bindSequence, bindTo, bindToProperty, bindToUtilities, Constant, Acquire

